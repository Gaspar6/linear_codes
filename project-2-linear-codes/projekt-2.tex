
\documentclass[10pt]{article} % use larger type; default would be 10pt

\usepackage[utf8]{inputenc} % set input encoding (not needed with XeLaTeX)
\usepackage[T1]{fontenc}
\usepackage{amsmath}
\usepackage{amsthm}
\usepackage{amssymb}
\usepackage[english]{babel}
\usepackage[margin=1in]{geometry}
\usepackage{verbatim}
\usepackage{wasysym}
\usepackage{fontawesome5}
\usepackage{graphicx}
\usepackage{caption}

\DeclareCaptionLabelFormat{image}{Rysunek #2}
\captionsetup[figure]{labelformat=image}

\setcounter{MaxMatrixCols}{11}



% STRONA 1 

\title{Algebra liniowa w analizie danych. Projekt 2. Kody liniowe.}
\author{Kacper Rodziewicz, Gaspar Sekula}
\date{10.05.2023}

\begin{document}
\maketitle

%%%%%%%%%%%%%%%%%%%%%%%%%%%%%%%%%%%%%%%%%%%%%%%%%%%%%%%%%%%%%%%%%%%%%%%%%%%%%%%%%%%%%%%%%%%%%%

\section*{Zadanko 1.}
\begin{proof}[Dowód] 
Wykażemy, że odległość Hamminga jest metryką.\\
Rozważmy funkcję $D: V \times V \rightarrow \mathbb{R}$ będącą odległością Hamminga, gdzie $V \subseteq \mathbb{K}^n$, $ \mathbb{K}^n$ jest przestrzenią liniową nad ciałem $\mathbb{K}$ i $V$ jest niepustym zbiorem.\\
To, że pierwsze dwa warunki są spełnione, jest oczywiste. Jednak gwoli profesjonalizmu, wyjaśnimy tę oczywistość. Weźmy dowolne $u, v, w \in V$. \\
\begin{enumerate}
\item[(1)] Równość $D(u, v) = 0$ zachodzi wtedy i tylko wtedy, gdy wszystkie współrzędne wektorów $u$ i $v$ są równe, co oznacza, iż $u=v$.
\item[(2)] Symetria wynika bezpośrednio z równoważności: $u$ różni się od $v$ o $k$ współrzędnych wtedy i tylko wtedy, gdy $v$ różni się od $u$ o $k$ współrzędnych.
\item[(3)] Wykażemy, że $D(u,w) \leq D(u,v) +D(v,w)$. Niech $i \in [n]$. Możliwe są dwa przypadki:
\begin{enumerate}
\item[(\faCrow)] $u_i \neq w_i$ Możliwe są trzy przypadki:
\begin{enumerate}
\item[(\faDiceOne)] $u_i \neq v_i \wedge v_i \neq w_i$
\item[(\faDiceTwo)]  $u_i = v_i \wedge v_i \neq w_i$
\item[(\faDiceThree)]  $u_i \neq v_i \wedge v_i = w_i$
\end{enumerate}
Przypadek  $u_i = v_i \wedge v_i = w_i$ nie zachodzi, bowiem wówczas $u_i = w_i$.
\item[(\faBone)] $u_i = w_i$ Możliwe są dwa przypadki:
\begin{enumerate}
\item[(\faDiceOne)] $u_i = v_i \wedge v_i = w_i$
\item[(\faDiceTwo)]  $u_i \neq v_i \wedge v_i \neq w_i$
\end{enumerate}
Przypadki $u_i = v_i \wedge v_i \neq w_i$ oraz $u_i \neq v_i \wedge v_i = w_i$ nie zachodzą, gdyż $u_i = w_i$. 
\end{enumerate}
Zauważmy, że licząc odległość Hamminga, iterujemy po współrzędnych od 1 do $n$. Gdy wartość $D(u,w)$ rośnie o 1, to wartość $D(u,v) +D(v,w)$ rośnie o 1 lub 2, podobnie gdy wartość $D(u,w)$ nie zmienia się, to wartość $D(u,v) +D(v,w)$ rośnie o 0 lub 2. Z tych obserwacji wynika, że $D(u,w) \leq D(u,v) +D(v,w)$.
\end{enumerate}
Toteż z (1), (2) i (3) odległość Hamminga jest metryką.\\
\end{proof}

%%%%%%%%%%%%%%%%%%%%%%%%%%%%%%%%%%%%%%%%%%%%%%%%%%%%%%%%%%%%%%%%%%%%%%%%%%%%%%%%%%%%%%%%%%%%%%

\newpage
\section*{Zadanko 2.}
\begin{proof}[Dowód] 
Niech $\mathcal{B} = (e_1, e_2, ..., e_k)$ będzie bazą $\mathcal{C}$, $\mathbb{G}$ będzie macierzą generującą (n,k)-kodu liniowego $\mathcal{C}$ nad $\mathbb{K}$, powstałą z bazy $\mathcal{B}$ oraz $v = \begin{pmatrix} a_1\\ a_2\\ \vdots \\ a_k\end{pmatrix}$, $a_i \in \mathbb{K}$, $i \in [k]$. Należy wykazać, że $ w =(v^T \cdot \mathbb{G})^T \in \mathcal{C}$.\\
Mamy więc $$ w =(v^T \cdot \mathbb{G})^T =  \mathbb{G}^T \cdot v = (e_1 | e_2 | ... | e_k) \cdot v = a_1\cdot e_1 + a_2\cdot e_2 + ... + a_k\cdot e_k,$$ gdzie $a_1\cdot e_1 + a_2\cdot e_2 + ... + a_k\cdot e_k$ jest kombinacją liniową wektorów z bazy $\mathcal{B}$, stąd $w \in \mathcal{C}$.\\
\end{proof}
%%%%%%%%%%%%%%%%%%%%%%%%%%%%%%%%%%%%%%%%%%%%%%%%%%%%%%%%%%%%%%%%%%%%%%%%%%%%%%%%%%%%%%%%%%%%%%

\newpage
\section*{Zadanko 3.}
\begin{proof}[Dowód]
Wykażemy, że dla dowolnego $(n,k)$-kodu liniowego $\mathcal{C}$ nad skończonym ciałem $\mathbb{K}$ i jego macierzy generującej $\mathbb{G}$ powstałej z bazy kodu $\mathcal{B}$ algorytm \verb|MinimizeHammingDistance| użyty do dekodowania słowa kodowego $w \in \mathcal{C}$ zwróci taki wektor $v \in \mathbb{K}^k$, który w wyniku zakodowania go z użyciem macierzy $\mathbb{G}$ da wektor $w$.\\ Ustalmy $w \in \mathcal{C}$. Zauważmy, że $|C| < |\mathbb{K}^k| = |\mathbb{K}|^k \in \mathbb{N}$, ponieważ $|\mathbb{K}|$ jest skończona. Znalezienie $m$ odbywa się w skończonej liczbie kroków, bowiem wystarczy przeiterować po każdym $v \in \mathcal{C}$ i liczyć $d(v,w)$. Szukane $m=0$, gdyż $w \in \mathcal{C}$ i znajdziemy $v \in \mathcal{C}$, takie że $d(v,w) = 0$ (innymi słowy, w zbiorze $\mathcal{C}$ znajdziemy $v=w$). \\
Drugi etap polega na wzięciu ze zbioru $L$ wektora $w$ (zbiór $L$ jest jednoelementowy, bowiem $d(v,w) = 0$ wtedy i tylko wtedy, gdy $v=w$). \\
Pozostało zapisać wektor $w$ w bazie $\mathcal{B}$, co, jak w poprzednich etapach, odbywa się w skończonej liczbie kroków. Niech  $\mathcal{B} = (e_1, e_2, ..., e_k)$ będzie bazą $\mathcal{C}$ ($e_i$ zapisujemy pionowo), $v = \begin{pmatrix} a_1\\a_2\\\vdots\\a_k\end{pmatrix}$ będzie zapisem współrzędnych wektora $w$ w bazie $\mathcal{B}$, $w = a_1\cdot e_1 + a_2\cdot e_2 + ... + a_k\cdot e_k$ oraz niech $\mathbb{G}$ będzie macierzą generującą kodu $\mathcal{C}$,\\ z definicji macierzy generującej kodu liniowego $\mathbb{G} =\begin{pmatrix} e_1^T\\e_2^T\\\vdots\\e_k^T\end{pmatrix} $. \\
Wówczas  $ (v^T \cdot \mathbb{G})^T =  \mathbb{G}^T \cdot v = a_1\cdot e_1 + a_2\cdot e_2 + ... + a_k\cdot e_k = w$.\\
\end{proof}
%%%%%%%%%%%%%%%%%%%%%%%%%%%%%%%%%%%%%%%%%%%%%%%%%%%%%%%%%%%%%%%%%%%%%%%%%%%%%%%%%%%%%%%%%%%%%%


\newpage
\section*{Zadanko 4.}
\begin{proof}[Dowód]
Ustalmy $u,v,x \in V$. Wykażemy, że $d(u,v) = d(u+x, v+x)$. Niech $d(u,v) = k$, gdzie $k \in [n] \cup \{0\}$, czyli wektory $u$ i $v$ różnią sie na $k$ indeksach. Dodawszy do $u$ i $v$ wektor $x$, wektory $u+x$ i $v+x$ wciąż różnią się na $k$ indeksach. Gwoli wyjaśnienia, dzieje się tak dlatego, że jeśli $u_i \neq v_i$ dla $i \in [n]$, to $u_i + x_i \neq v_i+x_i$; paralelnie jeśli $u_i = v_i$ dla $i \in [n]$, to $u_i + x_i = v_i+x_i$.\\
\end{proof}
%%%%%%%%%%%%%%%%%%%%%%%%%%%%%%%%%%%%%%%%%%%%%%%%%%%%%%%%%%%%%%%%%%%%%%%%%%%%%%%%%%%%%%%%%%%%%%


\newpage
\section*{Zadanko 5.}
Niech $u := (1,2,0,1)^T$, $v := (0,0,0,1)^T$. Łatwo zauważyć, że odległość Hamminga między wektorami $u$ i $v$ wynosi 2, sprawdzamy to w \textit{Mathematice} i rzeczywiście jest to 2. Niech $w_1 := (1,2,1,2,0)^T$, $w_2 := (1,1,1,1,1)^T$, $w_3 := (0,0,2,1,1)^T$, $w_4 := (2,2,2,1,0)^T$. Korzystamy z dobrze znanego programu \textit{Mathematica} i otrzymujemy:\\
$d(w_1, w_2) = 3$,\\
$d(w_1, w_3) = 5$,\\
$d(w_1, w_4) = 3$,\\
$d(w_2, w_3) = 3$,\\
$d(w_2, w_4) = 4$,\\
$d(w_3, w_4) = 3$,\\
wobec czego najbliżej w sensie Hamminga są $w_1$ i $w_2$, $w_1$ i $w_4$, $w_2$ i $w_3$ oraz $w_3$ i $w_4$.

%%%%%%%%%%%%%%%%%%%%%%%%%%%%%%%%%%%%%%%%%%%%%%%%%%%%%%%%%%%%%%%%%%%%%%%%%%%%%%%%%%%%%%%%%%%%%%


\newpage
\section*{Zadanko 6.}
Wygenerujemy wszystkie słowa kodowe dla (5, 3)-kodu liniowego $\mathcal{C}$ nad ciałem $\mathbb{Z}_7$, gdzie bazą kodu liniowego $\mathcal{C}$ jest  \vspace{10pt}
$$ \mathcal{B} = \begin{pmatrix} 
\begin{pmatrix} 1 \\0\\0\\2\\4 \end{pmatrix}, \begin{pmatrix} 0\\1\\0\\1\\0 \end{pmatrix}, \begin{pmatrix} 0\\0\\1\\5\\6 \end{pmatrix}
\end{pmatrix}.
$$ \vspace{10pt}
Naszym celem jest znalezienie wszystkich wektorów $v \in \mathbb{Z}_7^5$, które są kombinacjami liniowymi wektorów z bazy $\mathcal{B}$. Napisawszy prosty program w \textit{Pythonie}, otrzymujemy wszystkie słowa kodowe (jest ich 343): \\ \\
$(0, 0, 0, 0, 0)^T$, $(0, 0, 1, 5, 6)^T$, $(0, 0, 2, 3, 5)^T$, $(0, 0, 3, 1, 4)^T$, $(0, 0, 4, 6, 3)^T$, $(0, 0, 5, 4, 2)^T$, $(0, 0, 6, 2, 1)^T$, $(0, 1, 0, 1, 0)^T$, $(0, 1, 1, 6, 6)^T$, $(0, 1, 2, 4, 5)^T$, $(0, 1, 3, 2, 4)^T$, $(0, 1, 4, 0, 3)^T$, $(0, 1, 5, 5, 2)^T$, $(0, 1, 6, 3, 1)^T$, $(0, 2, 0, 2, 0)^T$, $(0, 2, 1, 0, 6)^T$, $(0, 2, 2, 5, 5)^T$, $(0, 2, 3, 3, 4)^T$, $(0, 2, 4, 1, 3)^T$, $(0, 2, 5, 6, 2)^T$, $(0, 2, 6, 4, 1)^T$, $(0, 3, 0, 3, 0)^T$, $(0, 3, 1, 1, 6)^T$, $(0, 3, 2, 6, 5)^T$, $(0, 3, 3, 4, 4)^T$, $(0, 3, 4, 2, 3)^T$, $(0, 3, 5, 0, 2)^T$, $(0, 3, 6, 5, 1)^T$, $(0, 4, 0, 4, 0)^T$, $(0, 4, 1, 2, 6)^T$, $(0, 4, 2, 0, 5)^T$, $(0, 4, 3, 5, 4)^T$, $(0, 4, 4, 3, 3)^T$, $(0, 4, 5, 1, 2)^T$, $(0, 4, 6, 6, 1)^T$, $(0, 5, 0, 5, 0)^T$, $(0, 5, 1, 3, 6)^T$, $(0, 5, 2, 1, 5)^T$, $(0, 5, 3, 6, 4)^T$, $(0, 5, 4, 4, 3)^T$, $(0, 5, 5, 2, 2)^T$, $(0, 5, 6, 0, 1)^T$, $(0, 6, 0, 6, 0)^T$, $(0, 6, 1, 4, 6)^T$, $(0, 6, 2, 2, 5)^T$, $(0, 6, 3, 0, 4)^T$, $(0, 6, 4, 5, 3)^T$, $(0, 6, 5, 3, 2)^T$, $(0, 6, 6, 1, 1)^T$, $(1, 0, 0, 2, 4)^T$, $(1, 0, 1, 0, 3)^T$, $(1, 0, 2, 5, 2)^T$, $(1, 0, 3, 3, 1)^T$, $(1, 0, 4, 1, 0)^T$, $(1, 0, 5, 6, 6)^T$, $(1, 0, 6, 4, 5)^T$, $(1, 1, 0, 3, 4)^T$, $(1, 1, 1, 1, 3)^T$, $(1, 1, 2, 6, 2)^T$, $(1, 1, 3, 4, 1)^T$, $(1, 1, 4, 2, 0)^T$, $(1, 1, 5, 0, 6)^T$, $(1, 1, 6, 5, 5)^T$, $(1, 2, 0, 4, 4)^T$, $(1, 2, 1, 2, 3)^T$, $(1, 2, 2, 0, 2)^T$, $(1, 2, 3, 5, 1)^T$, $(1, 2, 4, 3, 0)^T$, $(1, 2, 5, 1, 6)^T$, $(1, 2, 6, 6, 5)^T$, $(1, 3, 0, 5, 4)^T$, $(1, 3, 1, 3, 3)^T$, $(1, 3, 2, 1, 2)^T$, $(1, 3, 3, 6, 1)^T$, $(1, 3, 4, 4, 0)^T$, $(1, 3, 5, 2, 6)^T$, $(1, 3, 6, 0, 5)^T$, $(1, 4, 0, 6, 4)^T$, $(1, 4, 1, 4, 3)^T$, $(1, 4, 2, 2, 2)^T$, $(1, 4, 3, 0, 1)^T$, $(1, 4, 4, 5, 0)^T$, $(1, 4, 5, 3, 6)^T$, $(1, 4, 6, 1, 5)^T$, $(1, 5, 0, 0, 4)^T$, $(1, 5, 1, 5, 3)^T$, $(1, 5, 2, 3, 2)^T$, $(1, 5, 3, 1, 1)^T$, $(1, 5, 4, 6, 0)^T$, $(1, 5, 5, 4, 6)^T$, $(1, 5, 6, 2, 5)^T$, $(1, 6, 0, 1, 4)^T$, $(1, 6, 1, 6, 3)^T$, $(1, 6, 2, 4, 2)^T$, $(1, 6, 3, 2, 1)^T$, $(1, 6, 4, 0, 0)^T$, $(1, 6, 5, 5, 6)^T$, $(1, 6, 6, 3, 5)^T$, $(2, 0, 0, 4, 1)^T$, $(2, 0, 1, 2, 0)^T$, $(2, 0, 2, 0, 6)^T$, $(2, 0, 3, 5, 5)^T$, $(2, 0, 4, 3, 4)^T$, $(2, 0, 5, 1, 3)^T$, $(2, 0, 6, 6, 2)^T$, $(2, 1, 0, 5, 1)^T$, $(2, 1, 1, 3, 0)^T$, $(2, 1, 2, 1, 6)^T$, $(2, 1, 3, 6, 5)^T$, $(2, 1, 4, 4, 4)^T$, $(2, 1, 5, 2, 3)^T$, $(2, 1, 6, 0, 2)^T$, $(2, 2, 0, 6, 1)^T$, $(2, 2, 1, 4, 0)^T$, $(2, 2, 2, 2, 6)^T$, $(2, 2, 3, 0, 5)^T$, $(2, 2, 4, 5, 4)^T$, $(2, 2, 5, 3, 3)^T$, $(2, 2, 6, 1, 2)^T$, $(2, 3, 0, 0, 1)^T$, $(2, 3, 1, 5, 0)^T$, $(2, 3, 2, 3, 6)^T$, $(2, 3, 3, 1, 5)^T$, $(2, 3, 4, 6, 4)^T$, $(2, 3, 5, 4, 3)^T$, $(2, 3, 6, 2, 2)^T$, $(2, 4, 0, 1, 1)^T$, $(2, 4, 1, 6, 0)^T$, $(2, 4, 2, 4, 6)^T$, $(2, 4, 3, 2, 5)^T$, $(2, 4, 4, 0, 4)^T$, $(2, 4, 5, 5, 3)^T$, $(2, 4, 6, 3, 2)^T$, $(2, 5, 0, 2, 1)^T$, $(2, 5, 1, 0, 0)^T$, $(2, 5, 2, 5, 6)^T$, $(2, 5, 3, 3, 5)^T$, $(2, 5, 4, 1, 4)^T$, $(2, 5, 5, 6, 3)^T$, $(2, 5, 6, 4, 2)^T$, $(2, 6, 0, 3, 1)^T$, $(2, 6, 1, 1, 0)^T$, $(2, 6, 2, 6, 6)^T$, $(2, 6, 3, 4, 5)^T$, $(2, 6, 4, 2, 4)^T$, $(2, 6, 5, 0, 3)^T$, $(2, 6, 6, 5, 2)^T$, $(3, 0, 0, 6, 5)^T$, $(3, 0, 1, 4, 4)^T$, $(3, 0, 2, 2, 3)^T$, $(3, 0, 3, 0, 2)^T$, $(3, 0, 4, 5, 1)^T$, $(3, 0, 5, 3, 0)^T$, $(3, 0, 6, 1, 6)^T$, $(3, 1, 0, 0, 5)^T$, $(3, 1, 1, 5, 4)^T$, $(3, 1, 2, 3, 3)^T$, $(3, 1, 3, 1, 2)^T$, $(3, 1, 4, 6, 1)^T$, $(3, 1, 5, 4, 0)^T$, $(3, 1, 6, 2, 6)^T$, $(3, 2, 0, 1, 5)^T$, $(3, 2, 1, 6, 4)^T$, $(3, 2, 2, 4, 3)^T$, $(3, 2, 3, 2, 2)^T$, $(3, 2, 4, 0, 1)^T$, $(3, 2, 5, 5, 0)^T$, $(3, 2, 6, 3, 6)^T$, $(3, 3, 0, 2, 5)^T$, $(3, 3, 1, 0, 4)^T$, $(3, 3, 2, 5, 3)^T$, $(3, 3, 3, 3, 2)^T$, $(3, 3, 4, 1, 1)^T$, $(3, 3, 5, 6, 0)^T$, $(3, 3, 6, 4, 6)^T$, $(3, 4, 0, 3, 5)^T$, $(3, 4, 1, 1, 4)^T$, $(3, 4, 2, 6, 3)^T$, $(3, 4, 3, 4, 2)^T$, $(3, 4, 4, 2, 1)^T$, $(3, 4, 5, 0, 0)^T$, $(3, 4, 6, 5, 6)^T$, $(3, 5, 0, 4, 5)^T$, $(3, 5, 1, 2, 4)^T$, $(3, 5, 2, 0, 3)^T$, $(3, 5, 3, 5, 2)^T$, $(3, 5, 4, 3, 1)^T$, $(3, 5, 5, 1, 0)^T$, $(3, 5, 6, 6, 6)^T$, $(3, 6, 0, 5, 5)^T$, $(3, 6, 1, 3, 4)^T$, $(3, 6, 2, 1, 3)^T$, $(3, 6, 3, 6, 2)^T$, $(3, 6, 4, 4, 1)^T$, $(3, 6, 5, 2, 0)^T$, $(3, 6, 6, 0, 6)^T$, $(4, 0, 0, 1, 2)^T$, $(4, 0, 1, 6, 1)^T$, $(4, 0, 2, 4, 0)^T$, $(4, 0, 3, 2, 6)^T$, $(4, 0, 4, 0, 5)^T$, $(4, 0, 5, 5, 4)^T$, $(4, 0, 6, 3, 3)^T$, $(4, 1, 0, 2, 2)^T$, $(4, 1, 1, 0, 1)^T$, $(4, 1, 2, 5, 0)^T$, $(4, 1, 3, 3, 6)^T$, $(4, 1, 4, 1, 5)^T$, $(4, 1, 5, 6, 4)^T$, $(4, 1, 6, 4, 3)^T$, $(4, 2, 0, 3, 2)^T$, $(4, 2, 1, 1, 1)^T$, $(4, 2, 2, 6, 0)^T$, $(4, 2, 3, 4, 6)^T$, $(4, 2, 4, 2, 5)^T$, $(4, 2, 5, 0, 4)^T$, $(4, 2, 6, 5, 3)^T$, $(4, 3, 0, 4, 2)^T$, $(4, 3, 1, 2, 1)^T$, $(4, 3, 2, 0, 0)^T$, $(4, 3, 3, 5, 6)^T$, $(4, 3, 4, 3, 5)^T$, $(4, 3, 5, 1, 4)^T$, $(4, 3, 6, 6, 3)^T$, $(4, 4, 0, 5, 2)^T$, $(4, 4, 1, 3, 1)^T$, $(4, 4, 2, 1, 0)^T$, $(4, 4, 3, 6, 6)^T$, $(4, 4, 4, 4, 5)^T$, $(4, 4, 5, 2, 4)^T$, $(4, 4, 6, 0, 3)^T$, $(4, 5, 0, 6, 2)^T$, $(4, 5, 1, 4, 1)^T$, $(4, 5, 2, 2, 0)^T$, $(4, 5, 3, 0, 6)^T$, $(4, 5, 4, 5, 5)^T$, $(4, 5, 5, 3, 4)^T$, $(4, 5, 6, 1, 3)^T$, $(4, 6, 0, 0, 2)^T$, $(4, 6, 1, 5, 1)^T$, $(4, 6, 2, 3, 0)^T$, $(4, 6, 3, 1, 6)^T$, $(4, 6, 4, 6, 5)^T$, $(4, 6, 5, 4, 4)^T$, $(4, 6, 6, 2, 3)^T$, $(5, 0, 0, 3, 6)^T$, $(5, 0, 1, 1, 5)^T$, $(5, 0, 2, 6, 4)^T$, $(5, 0, 3, 4, 3)^T$, $(5, 0, 4, 2, 2)^T$, $(5, 0, 5, 0, 1)^T$, $(5, 0, 6, 5, 0)^T$, $(5, 1, 0, 4, 6)^T$, $(5, 1, 1, 2, 5)^T$, $(5, 1, 2, 0, 4)^T$, $(5, 1, 3, 5, 3)^T$, $(5, 1, 4, 3, 2)^T$, $(5, 1, 5, 1, 1)^T$, $(5, 1, 6, 6, 0)^T$, $(5, 2, 0, 5, 6)^T$, $(5, 2, 1, 3, 5)^T$, $(5, 2, 2, 1, 4)^T$, $(5, 2, 3, 6, 3)^T$, $(5, 2, 4, 4, 2)^T$, $(5, 2, 5, 2, 1)^T$, $(5, 2, 6, 0, 0)^T$, $(5, 3, 0, 6, 6)^T$, $(5, 3, 1, 4, 5)^T$, $(5, 3, 2, 2, 4)^T$, $(5, 3, 3, 0, 3)^T$, $(5, 3, 4, 5, 2)^T$, $(5, 3, 5, 3, 1)^T$, $(5, 3, 6, 1, 0)^T$, $(5, 4, 0, 0, 6)^T$, $(5, 4, 1, 5, 5)^T$, $(5, 4, 2, 3, 4)^T$, $(5, 4, 3, 1, 3)^T$, $(5, 4, 4, 6, 2)^T$, $(5, 4, 5, 4, 1)^T$, $(5, 4, 6, 2, 0)^T$, $(5, 5, 0, 1, 6)^T$, $(5, 5, 1, 6, 5)^T$, $(5, 5, 2, 4, 4)^T$, $(5, 5, 3, 2, 3)^T$, $(5, 5, 4, 0, 2)^T$, $(5, 5, 5, 5, 1)^T$, $(5, 5, 6, 3, 0)^T$, $(5, 6, 0, 2, 6)^T$, $(5, 6, 1, 0, 5)^T$, $(5, 6, 2, 5, 4)^T$, $(5, 6, 3, 3, 3)^T$, $(5, 6, 4, 1, 2)^T$, $(5, 6, 5, 6, 1)^T$, $(5, 6, 6, 4, 0)^T$, $(6, 0, 0, 5, 3)^T$, $(6, 0, 1, 3, 2)^T$, $(6, 0, 2, 1, 1)^T$, $(6, 0, 3, 6, 0)^T$, $(6, 0, 4, 4, 6)^T$, $(6, 0, 5, 2, 5)^T$, $(6, 0, 6, 0, 4)^T$, $(6, 1, 0, 6, 3)^T$, $(6, 1, 1, 4, 2)^T$, $(6, 1, 2, 2, 1)^T$, $(6, 1, 3, 0, 0)^T$, $(6, 1, 4, 5, 6)^T$, $(6, 1, 5, 3, 5)^T$, $(6, 1, 6, 1, 4)^T$, $(6, 2, 0, 0, 3)^T$, $(6, 2, 1, 5, 2)^T$, $(6, 2, 2, 3, 1)^T$, $(6, 2, 3, 1, 0)^T$, $(6, 2, 4, 6, 6)^T$, $(6, 2, 5, 4, 5)^T$, $(6, 2, 6, 2, 4)^T$, $(6, 3, 0, 1, 3)^T$, $(6, 3, 1, 6, 2)^T$, $(6, 3, 2, 4, 1)^T$, $(6, 3, 3, 2, 0)^T$, $(6, 3, 4, 0, 6)^T$, $(6, 3, 5, 5, 5)^T$, $(6, 3, 6, 3, 4)^T$, $(6, 4, 0, 2, 3)^T$, $(6, 4, 1, 0, 2)^T$, $(6, 4, 2, 5, 1)^T$, $(6, 4, 3, 3, 0)^T$, $(6, 4, 4, 1, 6)^T$, $(6, 4, 5, 6, 5)^T$, $(6, 4, 6, 4, 4)^T$, $(6, 5, 0, 3, 3)^T$, $(6, 5, 1, 1, 2)^T$, $(6, 5, 2, 6, 1)^T$, $(6, 5, 3, 4, 0)^T$, $(6, 5, 4, 2, 6)^T$, $(6, 5, 5, 0, 5)^T$, $(6, 5, 6, 5, 4)^T$, $(6, 6, 0, 4, 3)^T$, $(6, 6, 1, 2, 2)^T$, $(6, 6, 2, 0, 1)^T$, $(6, 6, 3, 5, 0)^T$, $(6, 6, 4, 3, 6)^T$, $(6, 6, 5, 1, 5)^T$, $(6, 6, 6, 6, 4)^T$.

%%%%%%%%%%%%%%%%%%%%%%%%%%%%%%%%%%%%%%%%%%%%%%%%%%%%%%%%%%%%%%%%%%%%%%%%%%%%%%%%%%%%%%%%%%%%%%



\newpage
\section*{Zadanko 7.}
Nietrudno zauważyć, że macierz generująca $\mathbb{G} = \begin{pmatrix} 1&0&0&2&4 \\ 0&1&0&1&0 \\ 0&0&1&5&6 \end{pmatrix}$. Niech $v = (2, 1, 3, 6, 0)^T$, oczywiście $v \in \mathbb{Z}_7^5$. Zaimplementowaliśmy algorytm \verb|MinimizeHammingDistance| w \textit{Pythonie}, wykorzystując bibliotekę \verb|scipy.spatial.distance| i otrzymaliśmy, że wektor $r$ współrzędnych wektora $w$ w bazie $\mathcal{B}$ to $r=(2,1,3)$. Sprawdzamy "manualnie" zgodność z prawdą i otrzymujemy pozytywny wynik \faSmile[regular].

%%%%%%%%%%%%%%%%%%%%%%%%%%%%%%%%%%%%%%%%%%%%%%%%%%%%%%%%%%%%%%%%%%%%%%%%%%%%%%%%%%%%%%%%%%%%%%

\newpage
\section*{Zadanko 8.}
W celu symulacji zaimplementowaliśmy odpowiedni kod w \textit{Pythonie}, obliczenia wykonujemy dla stałego ziarna generatora liczb losowych $random.seed = 2137$.
\begin{enumerate}

\item[(a)] Losowo wygenerowana macierz o 10 kolumnach i 4 wierszach, o elementach z ciała $\mathbb{Z}_5$ to $$A = \begin{pmatrix} 4&2&2&4&2&0&1&4&3&2\\
3&1&0&1&1&1&4&2&2&1\\
0&4&4&3&0&2&1&0&0&3\\
0&0&4&2&1&4&2&4&3&0\\
\end{pmatrix}.$$

\item[(b)] Normujemy macierz $A$ do przedziału $[0,1]$, dzieląc każdy jej element przez 4. Otrzymujemy:
$$X = \begin{pmatrix} \vspace{5pt} 1&\frac{1}{2}&\frac{1}{2}&1&\frac{1}{2}&0&\frac{1}{4}&1&\frac{3}{4}&\frac{1}{2}\\ \vspace{5pt}
\frac{3}{4}&\frac{1}{4}&0&\frac{1}{4}&\frac{1}{4}&\frac{1}{4}&1&\frac{1}{2}&\frac{1}{2}&\frac{1}{4}\\ \vspace{5pt}
0&1&1&\frac{3}{4}&0&\frac{1}{2}&\frac{1}{4}&0&0&\frac{3}{4}\\
0&0&1&\frac{1}{2}&\frac{1}{4}&1&\frac{1}{2}&1&\frac{3}{4}&0\\ \end{pmatrix}.$$ 
W celu stworzenia obrazu macierzy $X$, korzytsamy z funkcji \verb|Image| w Mathematice, otrzymujemy:\\

\begin{figure}[h]
	\centering
    \includegraphics[width=0.5\textwidth]{image.png}
    \caption{Obraz macierzy $X$.}
    \label{fig:example}
\end{figure}

\item[(c)] \begin{proof}[Dowód] 
Wykażemy, że istnieje (11,4)-kod liniowy $\mathcal{C}$ nad ciałem $\mathbb{Z}_6$, \vspace{10pt}
\\taki że $\mathbb{G} = \begin{pmatrix} 1&0&0&0&0&4&4&2&0&1&1\\
0&1&0&0&0&3&0&2&2&1&0\\
0&0&1&0&0&2&0&1&1&1&1\\
0&0&0&1&1&0&0&0&4&3&0\\
\end{pmatrix}$ jest macierzą generującą kodu $\mathcal{C}$.\vspace{10pt} \\
Niech $v_i$ będzie i-tym wektorem macierzy $\mathbb{G}$, $i \in [4]$ oraz $\mathcal{C} = \mathcal{L}(v_1, v_2, v_3, v_4)$ (nad ciałem $\mathbb{Z}_5)$. Zbiór $(v_1, v_2, v_3, v_4)$ jest liniowo niezależny, więc jest bazą $\mathcal{C}$. Ponadto $\mathcal{C} < \mathbb{Z}_5^{11}$, gdzie $\mathbb{Z}_5^{11}$ jest przestrzenią liniową nad ciałem $\mathbb{Z}_5$. Ciało $\mathbb{Z}_5$ ma skończoną liczność. Wobec tego $\mathcal{C}$ jest (11,4)-kodem liniowym nad ciałem $\mathbb{Z}_5$, $\mathcal{B}=(v_1, v_2, v_3, v_4)$ - bazą kodu $\mathcal{C}$, toteż $\mathbb{G}$ macierzą generującą kodu $\mathcal{C}$.\\

\end{proof}

\item[(d)]
Mamy daną macierz generującą $\mathbb{G} = \begin{pmatrix} 1&0&0&0&0&4&4&2&0&1&1\\
0&1&0&0&0&3&0&2&2&1&0\\
0&0&1&0&0&2&0&1&1&1&1\\
0&0&0&1&1&0&0&0&4&3&0\\
\end{pmatrix}$.\vspace{10pt} \\
 Niech $v$ := pierwsza kolumna macierzy $A$, czyli $v = (4, 3, 0,0 )^T$ oraz $w$ := kodowanie wektora $v$ przy użyciu macierzy generującej $\mathbb{G}$, $w = (v^T \cdot \mathbb{G})^T$. Oczywiście, każdy wektor musi mieć elementy z ciała $Z_5$, toteż działania wykonujemy modulo 5. Otrzymujemy $w = (4, 3, 0, 0, 0, 0, 1, 4, 1, 2, 4)^T$.
\newpage
\item[(e)]
Kodujemy każdą kolumnę $v_i$ ($i \in [10]$) macierzy $A$. Zapisujemy zakodowane wektory w macierzy $A'$, wówczas $A' = (w_1, w_2, ..., w_{10})$, gdzie $w_i = (v_i^T \cdot \mathbb{G})^T$, $i \in [10]$. Oczywiście, każdy wektor musi mieć elementy z ciała $Z_5$, toteż działania wykonujemy modulo 5. Mamy:
$$ A' = \begin{pmatrix} 4&2&2&4&2&0&1&4&3&2\\
3&1&0&1&1&1&4&2&2&1\\
0&4&4&3&0&2&1&0&0&3\\
0&0&4&2&1&4&2&4&3&0\\
0&0&4&2&1&4&2&4&3&0\\
0&4&1&0&1&2&3&2&3&2\\
1&3&3&1&3&0&4&1&2&3\\
4&0&3&3&1&4&1&2&0&4\\
1&1&0&3&1&0&2&0&1&0\\
2&2&3&4&1&0&2&3&4&1\\
4&1&1&2&2&2&2&4&3&0\\
\end{pmatrix} .$$
Dla każdego zakodowanego wektora (tj. kolumny macierzy) zasymulujemy wysłanie go do pewnego użytkownika poprzez kanał, który dla przesyłanego wektora $v$ dla każdej pozycji dodaje modulo 5 losową liczbę ze zbioru $\{0, 3\}$. Mamy więc $M$ := "przesłana" macierz $A'$:
$$ M = \begin{pmatrix} 4&2&2&4&2&0&1&2&3&2\\
3&1&3&1&1&1&4&2&2&1\\
0&4&2&3&0&2&1&0&0&3\\
0&0&4&2&4&4&2&4&3&0\\
0&0&4&2&1&4&2&4&3&0\\
3&4&1&0&1&2&1&2&3&2\\
1&3&3&1&3&0&4&1&2&3\\
4&0&3&3&1&4&1&2&0&4\\
1&1&0&3&1&0&2&0&1&0\\
0&2&3&4&4&0&2&3&4&1\\
4&1&1&2&0&2&2&4&3&0\\
\end{pmatrix} .$$

\item[(f, g)] Dla każdego zakodowanego wektora (każdej kolumny macierzy $B$) odkodowujemy przy użyciu algorytmu \verb|MinimizeHammingDistance|. Wyniki zapisujemy w macierzy $\mathcal{M}$ tak, że i-ta kolumna odpowiada i-temu wektorowi współrzędnych zakodowanego i przesłanego wektora $r$ współczynników wektora $w$ w bazie $\mathcal{B}$ (oznaczenia jak w algorytmie \verb|MinimizeHammingDistance|), \\gdzie $\mathcal{B} = ((1,0,0,0,0,4,4,2,0,1,1)^T,(
0,1,0,0,0,3,0,2,2,1,0)^T,(
0,0,1,0,0,2,0,1,1,1,1)^T,\\(
0,0,0,1,1,0,0,0,4,3,0)^T)$. Mamy:
$$ \mathcal{M} = \begin{pmatrix} 4&2&2&4&2&0&1&4&3&2\\3&1&0&1&1&1&4&2&2&1\\0&4&4&3&0&2&1&0&0&3\\0&0&4&2&1&4&2&4&3&0 \end{pmatrix}.$$ 

\item[(h)] Zauważamy, że \textbf{wszystkie} kolumny zostały dobrze odkodowane \faSmile[regular].

\newpage

\item[(i)] Normujemy macierz $\mathcal{M}$ do przedziału $[0,1]$, dzieląc każdy jej element przez 4. Otrzymujemy:
$$Y = \begin{pmatrix} \vspace{5pt} 1&\frac{1}{2}&\frac{1}{2}&1&\frac{1}{2}&0&\frac{1}{4}&1&\frac{3}{4}&\frac{1}{2}\\ \vspace{5pt}
\frac{3}{4}&\frac{1}{4}&0&\frac{1}{4}&\frac{1}{4}&\frac{1}{4}&1&\frac{1}{2}&\frac{1}{2}&\frac{1}{4}\\ \vspace{5pt}
0&1&1&\frac{3}{4}&0&\frac{1}{2}&\frac{1}{4}&0&0&\frac{3}{4}\\
0&0&1&\frac{1}{2}&\frac{1}{4}&1&\frac{1}{2}&1&\frac{3}{4}&0\\ \end{pmatrix}.$$
Tworzymy obraz macierzy $Y$, korzytsamy z funkcji \verb|Image| w Mathematice, otrzymujemy:\\

\begin{figure}[h]
	\centering
    \includegraphics[width=0.5\textwidth]{image.png}
    \caption{Obraz macierzy $Y$.}
    \label{fig:example}
\end{figure}

\end{enumerate}








\end{document}
